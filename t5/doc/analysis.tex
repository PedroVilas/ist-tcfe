\section{Theoretical Analysis}
\label{sec:analysis}

In this section it is made a theoretical analyses of the circuit shown in \textbf{Figure~\ref{fig:diagram_t5}}.
\begin{figure}[h] \centering
\includegraphics[width=0.95\linewidth]{diagram_t5.pdf}
%\vspace{-7cm}
\caption{Diagram of the circuit considered for the computations and simulations.}
\label{fig:diagram_t5}
\end{figure}


An OP-AMP BandPass filter (active bandpass filter) is used for, as the name suggests, block unwanted frequencies, only allowing a certain bandwith of frequencies to pass through while actively amplifing the signal for these frequencies. In this particular case, as refered in the introduction, we want the central frequency to be 1 KHz and the gain at this frequency to be 40 dB. This filter is not the same as the one used in a previous lab that only filtered the signal and did not amplify it (passive filter). For a passive filter we would only need resistors and capacitors, but for an active filter we also need transistors or OP-Amps in order to amplify the signal. In this case we have made use of the 741 OP-AMP in an non inverting manner. At the same time, this circuit is also quite different from the one of the previous lab assignment, in which we have had just the goal of optimizing a purpose-built audio amplifier (and had no intention of filtering the input signal).\par 


For the circuit shown in \textbf{Figure~\ref{fig:diagram_t5}} the lower cutoff frequency ($f_L = \frac{\omega_L}{2*\pi}$) is therefore:
\begin {equation}
	f_L= \frac{1}{R_1*C_1*2*\pi}   
	\label{eq:lcf}
\end{equation}


and the upper cutoff frequency ($f_H = \frac{\omega_H}{2*\pi}$) is given by: 

\begin {equation}
	f_H = \frac{1}{R_2 *C_2* 2*\pi}  
	\label{eq:ucf}
\end{equation}

and thus the central angular frequency $\omega_O$ is determined by geometric centre of these, as such given by:

\begin {equation}
	\omega_O= \sqrt{\omega_L * \omega_H }  
	\label{eq:CentralF}
\end{equation}
 
the gain is given by:

\begin {equation}
	Gain= |\frac{R_1*C_1*\omega*j}{1+R_1*C_1*\omega*j}*(1+\frac{R_3}{R_4})*\frac{1}{1+R_2*C_2*\omega*j}|   	
	\label{eq:gain}
\end{equation} 
the transfer function is given by: 

\begin {equation}
	T(s) = \frac{R_1*C_1*s}{1+R_1*C_1*s}*(1+\frac{R_3}{R_4})*\frac{1}{1+R_2*C_2*s}   	
	\label{eq:gain}
\end{equation} 

and the input and output impedances are given by: 

\begin {equation}
	Z_{in} = R_1 + \frac{1}{j*\omega_O*C_1} 
	\label{eq:impedances_in}
\end{equation}  

\begin {equation}
       Z_{out} = \frac{1}{j*\omega_O*C_2+\frac{1}{R2}}	
	\label{eq:impedances_out}
\end{equation}  

Using all that has been mentioned until now, we arrived at the final values considered for the circuit parameters (as named in the above figure) that maximized the merit. These ones are found in the following table. 

\hfill
 \parbox{1\linewidth}{
  \centering
  \begin{tabular}{|l|l|r|}
    \hline    
    {\bf Parameter} & {\bf Value} & {\bf Units }\\ \hline
    \input{values.tex}
  \label{tab:params}
  \end{tabular}
  }
\par

In the table, the results (output voltage gain in the passband, the central frequency, and the input and output impedances at this frequency) for both the theoretical analysis (on the right) and the Ngspice simulation (on the left) are presented in order for us to compare them easily. The final cost and merit values are also included.\par
The circuit used for the theoretical calculations was the one schematized above. This was also the main setup used for the Ngspice simulations, but a different setup was required to measure the output impedance (we'll get there in the next section).\par
As can be easily noted, the theoretical and simulation results differ a lot. This is due mainly (as would be expected by now) to the fact that the OPAMP model used includes capacitors, which introduces two extra poles in the transfer function (this subverts the bode plots obtained by simulation and theoretically, as will be seen). This also affects the upper cutoff frequency, and consequently the central passband frequency: while, in theory, the parameters used would not give us the desired frequency (in fact it would be around 100Hz and not around 1kHz), in practice, they do. \par
At the same time, the output impedance values differ by several orders of magnitude, and the gain obtained via both methods is off by approximately 50\% (we focussed on the simulation values in this process, thinking of them as what we wished to optimize) and in fact, the only similar values are the lower cutoff frequencies (and, obviously, the cost is the same).\par
As a result of all these discrepancies, the merit figures are very different: the simulation merit is about 100x greater than the one obtained from the theoretical calculations. Even so, the merit figure is very low, in comparison with previous laboratory assignments. This also occurs in our perspective as a result of different characteristics of the circuits used (in particular, given the high, fixed and unavoidable cost of the OPAMP and the limited components, which make it hard to achieve low enough deviations, essential for a seemingly high merit figure). This goes to show us that, for complex components such as these, a bad model or an incomplete model is worse than no model at all.\par

\hfill
 \parbox{1\linewidth}{
  \centering
  \begin{tabular}{|l|l|l|r|}
    \hline    
    {\bf Parameter} & {\bf Simulation} & {\bf Theoretical } & {\bf Units }\\ \hline
    $Zi_{total}$ & 385.039 & 283.57 & Ohm\\ \hline
$Zo_{total}$ & 4.00907 & 2.8721 & Ohm\\ \hline
$Zi_{gain}$ & - & 283.57 & Ohm\\ \hline
$Zo_{gain}$ & - & 191.47 & Ohm\\ \hline
$Zi_{output}$ & - & 31066.9 & Ohm\\ \hline
$Zo_{output}$ & - & 2.0552 & Ohm\\ \hline
Cost & 6090.8 & 6090.8 & MU\\ \hline
uco & 4254308.000 & 4254308.000 & Hz\\ \hline
lco & 20.540 & 84.305 & Hz\\ \hline
Bandwidth & 4254287.460 & 4254223.695 & Hz\\ \hline
$Gain_{gainstage}$ & - & 87.937 & [adimensional]\\ \hline
$Gain{outputstage}$ & - & 0.985 & [adimensional]\\ \hline
$Gain{total}$ & 39.575 & 86.592 & [adimensional]\\ \hline
MERIT & 1345.8247 & 717.4149 & gold medals\\ \hline

  \label{tab:results}
  \end{tabular}
  }
  
  It is to be noticed that uco and lco stand for upper cutoff frequeny and lower cutoff frequency, respectively.
  The next figures shown present the frequency response that has been obtained in the theoretical analysis.


\begin{figure}[H] \centering
\includegraphics[width=0.6\linewidth]{teo_gain.eps}
\caption{Output voltage gain frequency response of the BandPass Filter Amplifier. Note the plateau around 40dB (linear gain of 100) which occurs in the vicinity of 1kHz}
\label{fig:gain_octa}
\end{figure}

\begin{figure}[H] \centering
\includegraphics[width=0.6\linewidth]{teo_phase.eps}
\caption{Phase response of the BandPass Filter Amplifier. As expected, given the two poles of the transfer funtion, the phase bode plot goes from 90 degrees to -90 degrees}
\label{fig:phase_octa}
\end{figure}



\pagebreak


