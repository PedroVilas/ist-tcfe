\section{Introdução}
\label{sec:introdução}
% state the learning objective

\par O objetivo deste laboratório é estudar um circuito composto por um total de 11 componentes, desde resistências a fontes de tensão e de corrente (independentes e linearmente dependentes). \par
O circuito tem 8 nós e 4 malhas, como pode ser observado na figura. Os nós foram numerados arbitrariamente e foi definido que o nó 0 está ligado à terra, tendo portanto tensão nula.\par
As fontes de tensão estão representadas pelas letras Va e Vc e as fontes de corrente por Id e Ib. Ambas, Ib e Vc são linearmente dependentes, a partir das respetivas expressões presentes na figura.

\begin{figure}[h] \centering
\includegraphics[width=0.9\linewidth]{lab1_circuit.pdf}
\caption{Circuíto em Análise}
\label{fig:lab1_circuit}
\end{figure}

Os valores das resistências, das constantes e das fontes dependentes são gerados por um script de python a partir do número de aluno 86361 e estão representados na seguinte tabela:

\begin{table}[h]
  \centering
  \begin{tabular}{|l|r|}
    \hline    
    {\bf Nome} & {\bf Valores obtidos no script de Python} \\ \hline
	R1 &  1.0407324334365136\\ \hline
	R2 &  2.0857867519728686\\ \hline
	R3 &  3.071199615896663  \\ \hline
	R4 &  4.041770723234123 \\ \hline
	R5 &  3.140660073703873\\ \hline
	R6 &  2.0936811661064763 \\ \hline
	R7 &  1.0443244500565343  \\ \hline
	Va &  5.194209863050843 \\ \hline
	Id &  1.0412616050274464 \\ \hline
	Kb &  7.021062278588699\\ \hline
	Kc &  8.137326206873837\\ 
	\hline

  \end{tabular}
  \caption{Os valores apresentados para as resistências (R) estão em kiloohm (kOhm); Para a fonte de tensão é utilizada a letra Va e estã expressa em Volt (V); Id representa a fonte de corrente independente, a qual está em  miliAmpere (mA). As constantes Kc e Kb estão representadas em kiloOhm e miliSiemens, respetivamente.}
  \label{tab:python_values}
\end{table}

Pretende-se, portanto, analisar o circuito utilizando não só um programa de simulação, como também aplicando 2 métodos teóricos. Na Secção 2, é feita a análise teórica onde são aplicados os métodos dos nós e das malhas. De seguida, na Secção 3, é utilizado o programa Ngspice para simular e analisar o circuito. Os métodos serão comparados na última secção do relatório, secção 4.




