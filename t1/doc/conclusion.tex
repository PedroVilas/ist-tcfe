\section{Conclusão}
\label{sec:conclusao}

O objetivo deste laboratório foi cumprido com sucesso. Os resultados obtidos utilizando os diferentes métodos teóricos e a simulação no programa Ngspice coincidiram com precisão total, o que era esperado no circuito analisado. Isto porque o mesmo é composto apenas por componentes lineares e, como tal, os resultados não devem sofrer alterações dependendo do método utilizado, tal como mencionado nas aulas. \par
Foi, portanto, um laboratório bastante útil para aplicar os conhecimentos teóricos obtidos nas aulas e perceber um pouco melhor o funcionamento deste tipo de circuitos. Para além disso, o laboratório foi também proveitoso para uma introdução às ferramentas utilizadas. Esta nova realidade obriga a uma adaptação constante e ferramentas como o Github, Ngspice e Latex podem ser bastante úteis para uma melhor iteração à distância. Foi também necessária uma integração com o sistema operativo Linux e os respetivos ficheiros Make, o que não foi propriamente fácil, mas que veio acrescentar conhecimento que poderá vir a ser útil no futuro. 

