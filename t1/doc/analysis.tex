\section{Análise Teórica}
\label{sec:analise_teorica}
Análise do circuito utilizando 2 processos teóricos: método das malhas e método do nós.
\subsection{Método das Malhas} 

Foram criadas 4 variáveis para a aplicação do método das malhas, a correntes das malhas. Esta variáveis estão identificadas pelas letras IA, IB,IC e ID, na figura apresentada abaixo, onde também podem ser observados os sentidos utilizados. Sabendo os valores destas correntes e aplicando a lei de Ohm, são calculados os valores das correntes em cada um dos componentes e as tensões nos nós. Para determinar as 4 correntes das malhas, foram utilizadas as equações apresentadas abaixo, que foram resolvidas com o auxílio do Octave. 

\begin {equation}
	R_1I_A + R_3(I_A+I_B) + R_4(I_A+I_C) = V_a
	\label{eq:malha1}
\end{equation}

\begin {equation}
	R_6I_A + R_7I_C + R_4(I_C+I_A) = K_cI_C
	\label{eq:malha2}
\end{equation}

\begin {equation}
	I_B = K_bR_3(I_A+I_B)
	\label{eq:malha3}
\end{equation}

\begin {equation}
	I_D = I_d
	\label{eq:malha4}
\end{equation}


\subsection{Método dos Nós} 

Na aplicação do método dos nós foram considerados 9 nós, ou seja foi adicionada uma fonte de corrente fictícia o nó 7 e a resistência 6, criando assim mais um nó. Isto, porque na simulação feita no Ngspice era necessário definir a corrente sobre a qual a fonte de tensão Vc depende. Para haver uma coerência entre as 2 análises, manteve-se esta fonte fictícia. Obtêm-se então 9 equações, 5 diretamente obtidas aplicando a Lei de Kirchoff para as correntes (KCL) em cada um dos nós não ligados a fontes de corrente, 2 obtidas fazendo a diferença potencial entre os terminais das fontes de tensão e, por fim, 2 definindo a voltagem no nó 0 como nula e igualando as voltagens V8 e V7. As equações são as seguintes:\par


\begin {equation}
	V_0 = 0
	\label{eq1}
\end{equation}
\begin {equation}
	V_8 = V_7
	\label{eqn}
\end{equation}
\begin {equation}
	\frac{V_2-V_4}{R_3} + \frac{V_2-V_3}{R_2} + \frac{V_2-V_1}{R_1} = 0
	\label{eq4}
\end{equation}
\begin {equation}
	\frac{V_1-V_2}{R_1} + \frac{V_0 - V_8}{R_6} + \frac{V_0 - V_4}{R_4} = 0
	\label{eq3}
\end{equation}
\begin {equation}
	\frac{V_5-V_4}{R_5} + K_b(V_2-V_4) = I_d
	\label{eq6}
\end{equation}
\begin {equation}
	\frac{V_7-V_6}{R_7} + \frac{V_7 - V_0}{R_6} = 0
	\label{eq7}
\end{equation}
\begin {equation}
	\frac{V_3-V_2}{R_2} = K_b(V_2-V_4) 
	\label{eq5}
\end{equation}
\begin {equation}
	V_1 - V_0 = V_a
	\label{eq8}
\end{equation}
\begin {equation}
	V_4 - V_6 = K_c \frac{V_0 - V_7}{R_6}
	\label{eq9}
\end{equation}

Os valores obtidos para as correntes em cada um dos componentes e para as voltagens em cada um dos nós estão representados na seguinte tabela:
 \pagebreak 
\begin{table}[h]
  \centering
  \begin{tabular}{|l|r|r|}
    \hline    
    {\bf Nome} & {\bf Método das Malhas} & {\bf Métodos dos Nós}\\ \hline
    \input{final_table}
  \end{tabular}
  \caption{As variáveis que representam correntes estão precedidadas pelo simbolo @ e expressas em miliampere (mA); As restantes variáveis representam tensões e estão expressas em Volt (V)}
  \label{tab:valores_teoricos}
\end{table}

Pode verificar-se resultados exatamente iguais entre os 2 métodos, tal como esperado.


