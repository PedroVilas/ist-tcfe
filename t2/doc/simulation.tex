\section{Simulation Analysis}
\label{sec:simulation}

Using the Ngspice software three different analysis will be performed. First an Operating Point analysis, then a Transient analysis and, finally, a Frequency Analysis. The required steps for the analysis computation are described below:

\begin{itemize}
	\item First to obtain the voltages in all nodes and the currents in all branches, an operating point simulation for $t<0$ is done;
	\item To guarantee the continuity in the capacitor's discharge the right boundary conditions must be computed, so a an operating point simulation for $V_s(0) = 0$, replacing the capacitor voltage source $V_x = V_6-V_8$ (Voltages obtained in the previous step for respective nodes 6 and 8), must be done.
	\item Using the boundary conditions V(6) and V(8) obtained previously, a transient analysis to obtain the natural response of the circuit
	\item using f = 1 kHz and {\it $V_s$} as given in Figure 7, a repetition of the step number three in order to obtain the total response in node 6.
	\item Finally, for a frequency range 0.1 Hz to 1 MHz, a simulation of the frequency response in node 6.
\end{itemize}
 
 \begin{figure}[H] \centering
\includegraphics[width=0.5\linewidth]{time_step.pdf}
\caption{Time step conditions}
\label{fig:time_step}
\end{figure}

\subsection{Operating Point Analysis for $t<0$}
Do to the way Ngspice operates and simulates the circuit, we needed to create a fictional voltage source with a null value between node 7 and resistor 6, so that we were able to define the dependecy current for the dependent voltage source {\it $V_d$}.\par 
In Figure 2, the circuit and nodes used for the simulation can be seen.\par 
\textbf The simulated operating point results for the circuit under analysis and for $t<0$ can be seen in the table represented below (Table 6). 

\begin{table}[h!]
  \centering
  \begin{tabular}{|l|r|}
    \hline    
    {\bf Name} & {\bf Value [mA or V]} \\ \hline
    \input{op_tab}
  \end{tabular}
  \caption{$t<0$ Operating Point. The symbol @ indicates a variable of type {\em current}
    which is expressed in miliAmpere; the rest of the variables are of type {\it voltage} and expressed in Volt.}
  \label{tab:op}
\end{table}

\pagebreak
\subsection{Operating Point Analysis for $t=0$}
In this section the circuit was simulated using an operating point analysis with $V_s(0) = 0$ and 
with the capacitor replaced by a voltage source {\it $V_x=V(6)-V(8)$} with these as obtained in the last step. This step was taken because we must compute the boundary conditions that guarantee continuity in the capacitor's discharge (such may imply that the boundary conditions differ from those computed for $t<0$). In other words $V(6)-V(8)$ needs to be a continuos function in time (in this case particularly from $t<0$ to $t=0$), as there can not be a energy discontinuity in the capacitor ($E_C=\frac{1}{2}CV^{2}$). However, that does not imply that that $V(6)$ and $V(8)$ are continuos functions in time.
In \textbf{Table~\ref{tab:opeq}} the simulation results are presented. 
\begin{table}[h!]
  \centering
  \begin{tabular}{|l|r|}
    \hline    
    {\bf Name} & {\bf Value [mA or V and Ohm]} \\ \hline
    \input{opeq_tab}
  \end{tabular} 
  \caption{Operating point for {\it $v_s(0)=0$}. A variable preceded by @ is of type {\em current}
    and expressed in miliAmpere; variables are of type {\it voltage} and expressed in
    Volt. The equivalent resistance is in Ohms}
  \label{tab:opeq}
\end{table}

\pagebreak
%ponto 3
 
\subsection{ Natural solution for $V_6$ using transient analysis}
In this section the natural response of the circuit in the interval [0,20] ms was studied using a transient analysis simulation. To do so the boundary conditions V(6) and V(8) obtained in the previous section were used, as well as the NgSpice directive \textit{.ic}. These values are being obtained from the previous simulations run, and not from the theoretical previsions. 
\par
\begin{figure}[H] \centering
\includegraphics[width=0.6\linewidth]{trans.pdf}
\caption{Simulated natural response of $V_6(t)$ in the interval [0,20] ms. The \textit{x axis} represents the time in miliseconds and the \textit{y axis} the Potencial in node 6  in Volts.  }
\label{fig:transient}
\end{figure}

\pagebreak
%ponto 4
\subsection{ Total solution for $V_6$ using transient analysis}

In this section the total response of $V_6$ (natural + forced) is simulated using transient analysis. This is done by repeating the previous section, but using {\it $V_s$} as given in \textbf{Figure~\ref{fig:time_step}} and f = 1kHz.\par
\begin{figure}[H] \centering
\includegraphics[width=0.6\linewidth]{transv5vs.pdf}
\caption{Simulated response of $V_{6}(t)$ and of the stimulus $V_{s}(t)$ as functions of time from [0,20] ms. The \textit{x axis} represents the time in miliseconds and the \textit{y axis} the Voltage in Volts.}
\label{fig:resp_total}
\end{figure}

%ponto 5
\pagebreak
\subsection{ Frequency response in node 6}
In this section the frequency response in node 6 is simulated for the frequency range from 0.1 Hz to 1 MHz. 
The reasons of how and why $V_{6}(t)$ and $V_{s}(t)$ differ have been coverd in \textbf{subsection ~\ref{ref}}.\par
\begin{figure}[H] \centering
\includegraphics[width=0.6\linewidth]{acm.pdf}
\caption{Magnitude of $V_s(f)$, $V_c(f)$  and of $V_6(f)$. The \textit{x axis} represents the frequency in Hz, using a logarithmic scale and the \textit{y axis} the magnitude in dB.}
\label{fig:Magnitude}
\end{figure}
\pagebreak
\par
\begin{figure}[H] \centering
\includegraphics[width=0.6\linewidth]{phase.pdf}
\caption{Phase of $V_s(f)$, $V_c(f)$ and of $V_6(f)$. The \textit{x axis} represents the frequency in HZ, using a logarithmic scale and the \textit{y axis} the phase in degrees.}  
\label{fig:phase}
\end{figure}

\pagebreak
