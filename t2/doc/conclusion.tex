\section{Conclusion} 
\label{sec:conclusion}

In conclusion of our laboratory, we can say that our objective has been achieved, as we analysed the circuit that was proposed to us, which contained various resistances, a capacitor, and a sinusoidal voltage source $v_s$ that varies in time. We have also analysed static, time and frequency, in a theoretical way, by using the Octave math tool, and by simulating the circuit using the Ngspice tool. By doing this, we have obtained results very similar in both ways, resulting in an error matching 0, in values obtained theoretically, and in few cases of the simulation, we obtained a very small and insignificant error, which we can consider to be 0.\par
We can conclude that the reason for this almost unexisting error, is the fact that the circuit proposed was pretty simple, having only one capacitor and some linear components, so we can attribute this small discrepancys to the model used in Ngspice for the capacitor, and to the transient boundary condition analysis. This means that the greater the complexity of the components in the circuit, the greater are the discrepancys to be expected between the simulated results, and the results obtained theoretically, due to the also greater complexity of the models used for simulating these components in Ngspice.
